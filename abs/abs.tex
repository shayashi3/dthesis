\chapter*{Abstract}
Quantum chromodynamics (QCD) predicts the quark deconfinement and the transition to strongly interacting matter, quark gluon plasma (QGP), at extremely high temperature and density. 
Relativistic heavy ion collisions are an unique tool to study the properties of QGP. 
Since the yield of $J/\psi$ is expected to decrease in QGP due to the Debye screening of color charges, $J/\psi$ suppression is  one of the strong signatures of QGP formation. 
The PHENIX experiment at the Relativistic Heavy Ion Collider (RHIC) in the Brookhaven National Laboratory (BNL) observed the strong suppression of $J/\psi$ production in Au-Au collisions at $\sqrt{s_{NN}}=200$ GeV. 
ALICE and CMS at the Large Hadron Collider (LHC) at the European Organization for Nuclear Research (CERN) also confirmed the suppression of $J/\psi$ production in Pb-Pb collisions at $\sqrt{s_{NN}}=$2.76 TeV. 
However, it is necessary to understand the space-time evolution of heavy-ion collisions to study the formation of the QGP.  At RHIC, non-negligible normal nuclear matter effects at the initial and final stages of collisions are observed in d+Au collisions at $\sqrt{s_{NN}} ~=$ 200 GeV. 
Therefore the evaluation of normal nuclear matter effects is needed to extracted QGP signals.
 
This thesis presents the measurement of the inclusive $J/\psi$ production in p-Pb collisions at $\sqrt{s_{NN}}=$5.02 TeV with the ALICE central barrel detector.
The main aim of this analysis is the investigation of the normal nuclear matter effects in relativistic heavy ion collisions. 

The measured nuclear modification factor ($R_{\rm{pPb}}$) of the inclusive $J/\psi$ production at mid-rapidity (-1.37 $<$ $y$ $<$ 0.43) show the clear suppression around 1.5-4.5 GeV/$c$. 
% compared to the model calculation including the modification of the nuclear parton distribution function (nPDF) due to the gluon shadowing. 
The coherent parton energy loss model including nPDF modification describes $y$ dependence of $R_{\rm{pPb}}$ qualitatively. 
It also show the consistency with data on $p_{t}$ dependence of $R_{\rm{pPb}}$ at mid-rapidity. 
%$p_{\rm{T}}$ dependence of this model is not excluded within the current statistics. 

The normal nuclear matter effects in heavy ion collisions is factorized by the product of $R_{\rm{pPb}}$ and the ratio of  the nuclear modification factor in Pb-Pb collision ($R_{AA}$) to the product of the measured $R_{\rm{pPb}}$ is calculated.
This comparison of results in p-Pb and Pb-Pb collisions indicates the clear suppression of high $p_{\rm{T}}$  $J/\psi$ production in Pb-Pb collisions. 
On the other hand, the enhancement of $J/\psi$ yield is observed at low $p_{\rm{T}}$ ($<$ 4.5 GeV/$c$). 
It suggests that the $J/\psi$ generation in QGP is same level as the initial $J/\psi$ production in heavy ion collisions at LHC energies. 

%The measured $R_{\rm{pPb}}$ is good tool for the evaluation of the initial stage effects in relativistic heavy ion collisions. 
%Compared to the expected nuclear modification from the measured $J/\psi$ $R_{\rm{pPb}}$, $J/\psi$ production in Pb-Pb collisions show the clear suppression at higher $p_{T}$. 
%On the other hand, an enhancement of $J/\psi$ production is observed at lower $p_{\rm{T}}$ in Pb-Pb collisions. 
%This is the clear evidence that regeneration of $J/\psi$ in QGP is dominant in low $p_{\rm{T}}$ and suppression by color screening is governed at high $p_{\rm{T}}$ in Pb-Pb collisions.
%Finally the initial energy density dependence of surviving $J/\psi$ and regenerated $J/\psi$ is shown. 
%Compared with the previous experiments, the sequential melting is observed for surviving $J/\psi$. 
%It strongly supports the color screening picture. 
%Enormous regeneration of $J/\psi$  suggest the charm quark thermalization and an evidence of strong coupled QGP. 
