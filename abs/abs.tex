\chapter*{Abstract}
%Quantum chromodynamics (QCD) predicts the quark deconfinement and the transition to strongly interacting matter, quark gluon plasma (QGP), at extremely high temperature and density. 
%Relativistic heavy ion collisions are an unique tool to study the properties of QGP. 
%Since the yield of $J/\psi$ is expected to decrease in QGP due to the Debye screening of color charges, $J/\psi$ suppression is  one of the strong signatures of QGP formation. 
%The PHENIX experiment at the Relativistic Heavy Ion Collider (RHIC) in the Brookhaven National Laboratory (BNL) observed the strong suppression of $J/\psi$ production in Au-Au collisions at $\sqrt{s_{NN}}=200$ GeV. 
%ALICE and CMS at the Large Hadron Collider (LHC) at the European Organization for Nuclear Research (CERN) also confirmed the suppression of $J/\psi$ production in Pb-Pb collisions at $\sqrt{s_{NN}}=$2.76 TeV. 
%However, it is necessary to understand the space-time evolution of heavy-ion collisions to study the formation of the QGP.  At RHIC, non-negligible normal nuclear matter effects at the initial and final stages of collisions are observed in d+Au collisions at $\sqrt{s_{NN}} ~=$ 200 GeV. 
%Therefore the evaluation of normal nuclear matter effects is needed to extracted QGP signals.
Quantum chromodynamics predicts quark deconfinement and the transition to strongly interacting matter, quark gluon plasma (QGP), at extremely high temperature and density. 
Since the dissociation of $J/\psi$ in the QGP is expected due to Debye screening of color charges, suppression of the $J/\psi$ yield is considered to be as one of the strong signatures of QGP formation. 

Relativistic heavy ion collisions are a unique tool for studying the properties of the QGP. 
%PHENIX at the Relativistic Heavy Ion Collider (RHIC) in the Brookhaven National Laboratory (BNL) observed the strong suppression of the $J/\psi$ yield in Au-Au collisions at $\sqrt{s_{NN}}=200$ GeV. 
Strong suppression of the $J/\psi$ yield was observed in Au--Au collisions at $\sqrt{s_{NN}}=200$ GeV by the PHENIX experiment in the Relativistic Heavy Ion Collider (RHIC) at Brookhaven National Laboratory.
The $J/\psi$ yields measured by the ALICE experiment at the Large Hadron Collider (LHC) in the European Organization for Nuclear Research (CERN) were also suppressed in Pb-Pb collisions at $\sqrt{s_{NN}}=$2.76 TeV. 
In addition, non-negligible suppression of the $J/\psi$ yield was observed in d--Au collisions at $\sqrt{s_{NN}}=$ 200 GeV at the RHIC.
Suppression of d--Au collisions is expected due to normal nuclear matter effects such as gluon shadowing and nuclear absorption.  
Since measurements of heavy ion collisions are also affected by these effects, their understanding in heavy ion collisions is essential in the discussion of the QGP effects in heavy ion collisions.
Normal nuclear matter effects are also expected to be relevant in heavy ion collisions at the LHC. 

This thesis presents the measurement of inclusive $J/\psi$ production in p--Pb collisions at $\sqrt{s_{NN}}=$ 5.02 TeV at the ALICE central barrel detector.
The main aim of this analysis is to investigate the normal nuclear matter effects on $J/\psi$ production in relativistic heavy ion collisions. 

The inclusive $J/\psi$ nuclear modification factor ($R_{\rm{pPb}}$) at mid-rapidity ($-1.37 <y<0.43$) was measured as a function of the transverse momentum $p_{\rm{T}}$. 
The results show significant suppression of the $J/\psi$ yield around 1.5--4.5 GeV/$c$. 
The coherent parton energy loss model qualitatively describes the dependence of the measured $R_{\rm{pPb}}$ on the rapidity $y$ and $p_{\rm{T}}$.  
Gluon shadowing calculations also show that the $y$ dependence of the measured $J/\psi$ $R_{\rm{pPb}}$ has similar features.
%It also show the consistency with data on $p_{t}$ dependence of $R_{\rm{pPb}}$ at mid-rapidity. 
%$p_{\rm{T}}$ dependence of this model is not excluded within the current statistics. 

Under the assumptions that gluon shadowing predominantly affects the $J/\psi$ yield in p--Pb collisions at the LHC, the nuclear modification factor associated with normal nuclear matter effects is approximated by the product of $R_{\rm{pPb}}$ in heavy ion collisions.
In order to estimate the QGP effects in Pb--Pb collisions at the LHC, the surviving fraction ($S_{AA$) is introduced. 
It is defined as the ratio of the nuclear modification factor in Pb--Pb collisions ($R_{AA}$) to the product of $R_{\rm{pPb}}$.
The measured $J/\psi$ $S_{AA}$ is significantly less than unity at high $p_{\rm{T}}$ in Pb--Pb collisions. 
This result is consistent with the color screening effect in the QGP. 
At low $p_{\rm{T}}$ ($<$4.5 GeV/$c$), clear enhancement of the $J/\psi$ $S_{AA}>1$ is observed. 
This enhancement suggests that a large amount of $J/\psi$ is regenerated in the QGP at LHC energies. 

%The measured $R_{\rm{pPb}}$ is good tool for the evaluation of the initial stage effects in relativistic heavy ion collisions. 
%Compared to the expected nuclear modification from the measured $J/\psi$ $R_{\rm{pPb}}$, $J/\psi$ production in Pb-Pb collisions show the clear suppression at higher $p_{T}$. 
%On the other hand, an enhancement of $J/\psi$ production is observed at lower $p_{\rm{T}}$ in Pb-Pb collisions. 
%This is the clear evidence that regeneration of $J/\psi$ in QGP is dominant in low $p_{\rm{T}}$ and suppression by color screening is governed at high $p_{\rm{T}}$ in Pb-Pb collisions.
%Finally the initial energy density dependence of surviving $J/\psi$ and regenerated $J/\psi$ is shown. 
%Compared with the previous experiments, the sequential melting is observed for surviving $J/\psi$. 
%It strongly supports the color screening picture. 
%Enormous regeneration of $J/\psi$  suggest the charm quark thermalization and an evidence of strong coupled QGP. 

