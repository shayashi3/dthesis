\chapter{Conclusion }
\label{chap_con}
$J/\psi$ suppression is one of the most strong evidence of quark-gluon plasma (QGP) formation in relativistic heavy ion collisions. 
The previous $J/\psi$ measurements at RHIC show the strong suppression of $J/\psi$ production in relativistic Au-Au collisions. 
The experimental investigation of $J/\psi$ suppression has also been performed at LHC and the suppression of $J/\psi$ production in Pb-Pb collision was observed. 
However the measured $J/\psi$ $R_{\rm{AA}}$ is affected by the not only QGP effects but also other nuclear matter effects such as the effect of the modification of the parton distribution function from free protons. 
 
In order to exploit the normal nuclear matter effects experimentally, 
the measurement of inclusive $J/\psi$ production in p-Pb collisions has been carried out at $\sqrt{s_{NN}}=$5.02 TeV with the ALICE detectors.
The data was collected in 2013 winter. 
$J/\psi$ was measured via dielectron decay channel in the mid-rapidity region $-1.37<y<0.43$. 
Electrons are reconstructed using information of ITS and TPC in the ALICE central barrel. 

The cross section of inclusive $J/\psi$ in p-Pb collision at $\sqrt{s_{NN}}=$5.02 TeV is determined by, 
\begin{equation}
  \frac{d\sigma_{J/\psi}}{dy} =   930 ~\pm 83 ~\rm{(stat)} ~\pm ~59 ~\rm{(uncorr. syst)} ~\pm~ 31\rm{(corr. syst)} ~\mu  \rm{b}
%   \frac{d\sigma_{J/\psi}}{dy} =  946 ~\pm  ~50.1 ~\rm{(stat)} ~\pm 63 ~\rm{(~syst)}~\mu \rm{b}
\end{equation}
The nuclear modification factor in p-Pb collisions ($R_{\rm{pPb}}$) is extracted by comparing the $J/\psi$ spectrum in p-Pb collisions to the interpolated pp results from the measured pp cross section at various energy. 
$R_{\rm{pPb}}$ of inclusive $J/\psi$ production at mid-rapidity is extracted as, 
\begin{equation}
 % R_{\rm{pPb}} = \rm{0.77 ~\pm ~0.04 ~(stat) ~\pm ~0.14 ~(~syst) }
 	  R_{\rm{pPb}} = \rm{0.74 ~\pm ~0.07 ~(stat) ~\pm ~0.13 ~(uncorr.~syst) ~\pm ~0.03 ~(corr.~syst) }
\end{equation}
The non-negligible suppression was observed due to the normal nuclear matter effects. 
Compared with the forward rapidity measurement via dimuon decay channel, the magnitudes of $R_{\rm{pPb}}$ are compatible within the uncertainties. 

The several model including the modification of nPDF, coherent energy loss, and gluon saturation with the CGC frame work. 
The coherent energy loss model shows the reasonable description of the both of $y$ and $p_{\rm{T}}$ dependence. 
On the other hand, although there is still model dependence of the suppression strength, $R_{g}^{Pb}$ calculated with several nPDFs shows the weak $x$ dependence at small $x$.
It is qualitatively consistent with the less rapidity dependence of the data between mid-rapidity and forward rapidity measurements.  
The expected $R_{g}^{Pb}$ under $2\rightarrow 1$ assumption is compatible to the calculation with the shadowing parametrization within the uncertainty. 
CGC calculations might describe the data at forward $J/\psi$ production. 
However the uncertainty is still large for both data and model calculation. 
The main sources of the uncertainties of the experimental data are the statistical uncertainty and the uncertainty of the interpolated pp reference spectra. 
To achieve high precision, the reduction of these uncertainties are needed.  

%The model calculation which includes the shadowing effects parametrized by EPS09 nPDF and color evaporation model (CEM) show the overestimate of the nuclear modification factor compared to the results in this thesis. 
%The coherent energy loss model describes the $y$ dependence of $R_{pPb}$ qualitatively although the it depends on the nPDF and $\hat{q}$ parametrization.
%One calculation based on the color glass condensate framework underestimates $J/\psi$ production at mid-rapidity and forward rapidity. 
%The $p_{\rm{T}}$ differential $R_{\rm{pPb}}$ was also extracted . 
%At 1.5 $<$ $p_{\rm{T}}$ $<$ 4.5 GeV/$c$, the suppression ($R_{pPb}\sim$ 0.6-0.7) is observed. 
%The $p_{T}$ dependence at mid-rapidity is also compared to the above model calculation.  
%%On the other hand, the clear enhancement and all compared model calculation don't show this behavior. 
%The most significant suppression is expected at lowest bins due to the modification of nPDF at relatively small $Q^{2}$. 
%The data does not constrain to model calculations strongly and the conclusive explanation of  $J/\psi$ production  in p-Pb collisions is still missing. 


Under the simple assumption, a factorization of normal nuclear matter effects in Pb-Pb collisions is performed from measured p-Pb data.  
We approximated the nuclear modification factor of the normal nuclear matter effects as the convolution of $R_{\rm{pPb}}$. 
Compared between ($R_{\rm{pPb}})^{2}$ at $\sqrt{s_{NN}}=$5.02 TeV and $R_{\rm{AA}}$ in Pb-Pb collisions at $\sqrt{s_{NN}}=$2.76 TeV,  
the suppression is seen at high $p_{T}$ above 4.5 GeV/$c$. 
On the other hand, a significant enhancement of $J/\psi$ yield above 1.5 is observed at lower $p_{\rm{T}}$.
This suggests newly-formed $J/\psi$ in QGP is expected to be dominant or compatible to the initial $J/\psi$ in heavy ion collisions at LHC.
This feature is qualitatively consistent with the color screening and regeneration pictures. 


%To evaluate the normal nuclear matter effect experimentally ,  a simple factorized approximation is performed.  
%We approximated the nuclear modification factor of the normal nuclear matter effects as the product of $R_{\rm{pPb}}$. 
%Compared between $R_{\rm{pPb}}^{2}$ at $\sqrt{s_{NN}}=$5.02 TeV and $R_{\rm{AA}}$ in Pb-Pb collisions at $\sqrt{s_{NN}}=$2.76 TeV,  
%the suppression was confirmed at high $p_{T}$ above 6.5 GeV/$c$. 
%On the other hand, an enhancement of $J/\psi$ yield was observed at lower $p_{\rm{T}}$.
%This is the clear evidence that regeneration of $J/\psi$ in QGP is dominant in low $p_{\rm{T}}$ and suppression by color screening is governed at high $p_{\rm{T}}$ in Pb-Pb collisions.
%In order to extract the thermodynamical features of $J/\psi$ production, the suppression strength and enhance factor of regeneration is compared by the previous experimental results. 
%The suppression surviving $J/\psi$ shows the same sequential trend which is consistent with color screening picture. 
%On the other hand, the regenerated $J/\psi$ shows the their enhance factor above unity. 
%This enormous enhancement suggests the full thermalization of charm quarks.
%By correcting the normal nuclear matter effects, hidden behaviors of $J/\psi$ production in heavy ion collisions shows up and the direct comparison of 


Run2 of LHC started in 2015 and it gives us the opportunity to investigate new collision energy in pp at $\sqrt{s}=$13, 14 TeV, p-Pb at $\sqrt{s_{NN}}=$8 TeV, Pb-Pb at $\sqrt{s_{NN}}=$5.02 TeV and improved statistics. 
Furthermore pp at $\sqrt{s}=$5.02 TeV is also scheduled as the reference measurement of p-Pb and Pb-Pb. 
This lead to the significant reduction of the systematic uncertainty of the pp reference. 
It is crucial for quantitative explanation of $J/\psi$ production in relativistic heavy ion collisions. 
